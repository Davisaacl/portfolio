% Options for packages loaded elsewhere
\PassOptionsToPackage{unicode}{hyperref}
\PassOptionsToPackage{hyphens}{url}
%
\documentclass[
]{article}
\usepackage{lmodern}
\usepackage{amssymb,amsmath}
\usepackage{ifxetex,ifluatex}
\ifnum 0\ifxetex 1\fi\ifluatex 1\fi=0 % if pdftex
  \usepackage[T1]{fontenc}
  \usepackage[utf8]{inputenc}
  \usepackage{textcomp} % provide euro and other symbols
\else % if luatex or xetex
  \usepackage{unicode-math}
  \defaultfontfeatures{Scale=MatchLowercase}
  \defaultfontfeatures[\rmfamily]{Ligatures=TeX,Scale=1}
\fi
% Use upquote if available, for straight quotes in verbatim environments
\IfFileExists{upquote.sty}{\usepackage{upquote}}{}
\IfFileExists{microtype.sty}{% use microtype if available
  \usepackage[]{microtype}
  \UseMicrotypeSet[protrusion]{basicmath} % disable protrusion for tt fonts
}{}
\makeatletter
\@ifundefined{KOMAClassName}{% if non-KOMA class
  \IfFileExists{parskip.sty}{%
    \usepackage{parskip}
  }{% else
    \setlength{\parindent}{0pt}
    \setlength{\parskip}{6pt plus 2pt minus 1pt}}
}{% if KOMA class
  \KOMAoptions{parskip=half}}
\makeatother
\usepackage{xcolor}
\IfFileExists{xurl.sty}{\usepackage{xurl}}{} % add URL line breaks if available
\IfFileExists{bookmark.sty}{\usepackage{bookmark}}{\usepackage{hyperref}}
\hypersetup{
  pdftitle={Parcial 2},
  pdfauthor={David López},
  hidelinks,
  pdfcreator={LaTeX via pandoc}}
\urlstyle{same} % disable monospaced font for URLs
\usepackage[margin=1in]{geometry}
\usepackage{color}
\usepackage{fancyvrb}
\newcommand{\VerbBar}{|}
\newcommand{\VERB}{\Verb[commandchars=\\\{\}]}
\DefineVerbatimEnvironment{Highlighting}{Verbatim}{commandchars=\\\{\}}
% Add ',fontsize=\small' for more characters per line
\usepackage{framed}
\definecolor{shadecolor}{RGB}{248,248,248}
\newenvironment{Shaded}{\begin{snugshade}}{\end{snugshade}}
\newcommand{\AlertTok}[1]{\textcolor[rgb]{0.94,0.16,0.16}{#1}}
\newcommand{\AnnotationTok}[1]{\textcolor[rgb]{0.56,0.35,0.01}{\textbf{\textit{#1}}}}
\newcommand{\AttributeTok}[1]{\textcolor[rgb]{0.77,0.63,0.00}{#1}}
\newcommand{\BaseNTok}[1]{\textcolor[rgb]{0.00,0.00,0.81}{#1}}
\newcommand{\BuiltInTok}[1]{#1}
\newcommand{\CharTok}[1]{\textcolor[rgb]{0.31,0.60,0.02}{#1}}
\newcommand{\CommentTok}[1]{\textcolor[rgb]{0.56,0.35,0.01}{\textit{#1}}}
\newcommand{\CommentVarTok}[1]{\textcolor[rgb]{0.56,0.35,0.01}{\textbf{\textit{#1}}}}
\newcommand{\ConstantTok}[1]{\textcolor[rgb]{0.00,0.00,0.00}{#1}}
\newcommand{\ControlFlowTok}[1]{\textcolor[rgb]{0.13,0.29,0.53}{\textbf{#1}}}
\newcommand{\DataTypeTok}[1]{\textcolor[rgb]{0.13,0.29,0.53}{#1}}
\newcommand{\DecValTok}[1]{\textcolor[rgb]{0.00,0.00,0.81}{#1}}
\newcommand{\DocumentationTok}[1]{\textcolor[rgb]{0.56,0.35,0.01}{\textbf{\textit{#1}}}}
\newcommand{\ErrorTok}[1]{\textcolor[rgb]{0.64,0.00,0.00}{\textbf{#1}}}
\newcommand{\ExtensionTok}[1]{#1}
\newcommand{\FloatTok}[1]{\textcolor[rgb]{0.00,0.00,0.81}{#1}}
\newcommand{\FunctionTok}[1]{\textcolor[rgb]{0.00,0.00,0.00}{#1}}
\newcommand{\ImportTok}[1]{#1}
\newcommand{\InformationTok}[1]{\textcolor[rgb]{0.56,0.35,0.01}{\textbf{\textit{#1}}}}
\newcommand{\KeywordTok}[1]{\textcolor[rgb]{0.13,0.29,0.53}{\textbf{#1}}}
\newcommand{\NormalTok}[1]{#1}
\newcommand{\OperatorTok}[1]{\textcolor[rgb]{0.81,0.36,0.00}{\textbf{#1}}}
\newcommand{\OtherTok}[1]{\textcolor[rgb]{0.56,0.35,0.01}{#1}}
\newcommand{\PreprocessorTok}[1]{\textcolor[rgb]{0.56,0.35,0.01}{\textit{#1}}}
\newcommand{\RegionMarkerTok}[1]{#1}
\newcommand{\SpecialCharTok}[1]{\textcolor[rgb]{0.00,0.00,0.00}{#1}}
\newcommand{\SpecialStringTok}[1]{\textcolor[rgb]{0.31,0.60,0.02}{#1}}
\newcommand{\StringTok}[1]{\textcolor[rgb]{0.31,0.60,0.02}{#1}}
\newcommand{\VariableTok}[1]{\textcolor[rgb]{0.00,0.00,0.00}{#1}}
\newcommand{\VerbatimStringTok}[1]{\textcolor[rgb]{0.31,0.60,0.02}{#1}}
\newcommand{\WarningTok}[1]{\textcolor[rgb]{0.56,0.35,0.01}{\textbf{\textit{#1}}}}
\usepackage{graphicx,grffile}
\makeatletter
\def\maxwidth{\ifdim\Gin@nat@width>\linewidth\linewidth\else\Gin@nat@width\fi}
\def\maxheight{\ifdim\Gin@nat@height>\textheight\textheight\else\Gin@nat@height\fi}
\makeatother
% Scale images if necessary, so that they will not overflow the page
% margins by default, and it is still possible to overwrite the defaults
% using explicit options in \includegraphics[width, height, ...]{}
\setkeys{Gin}{width=\maxwidth,height=\maxheight,keepaspectratio}
% Set default figure placement to htbp
\makeatletter
\def\fps@figure{htbp}
\makeatother
\setlength{\emergencystretch}{3em} % prevent overfull lines
\providecommand{\tightlist}{%
  \setlength{\itemsep}{0pt}\setlength{\parskip}{0pt}}
\setcounter{secnumdepth}{-\maxdimen} % remove section numbering

\title{Parcial 2}
\author{David López}
\date{28/10/2020}

\begin{document}
\maketitle

Al inicio se presentan todas las preguntas que requirieron el uso de R
para su interpretación, desarrollo y solución.

\hypertarget{pregunta-1}{%
\subsection{Pregunta 1}\label{pregunta-1}}

En esta pregunta modelo una simulación de variables aleatorias con
función de densidad Rayleigh. La hago por medio de variables antiéticas
para minimizar la varianza producida en la simulación. Utilizo el
siguiente modelo de variables antiéticas: \(\frac{X+Y}{2}\)Entonces,
para ver los resultados presento un histograma de los valores obtenidos
a partir de la simulación. Para ello utilizo los siguientes valores:
\(n=10000\) iteraciones y de parámetro de escala de la función de
densidad de Rayleigh \(\sigma =10\).

\begin{Shaded}
\begin{Highlighting}[]
\CommentTok{# Pregunta 1 parte 1}
\KeywordTok{library}\NormalTok{(VGAM)}
\end{Highlighting}
\end{Shaded}

\begin{verbatim}
## Warning: package 'VGAM' was built under R version 3.6.3
\end{verbatim}

\begin{verbatim}
## Loading required package: stats4
\end{verbatim}

\begin{verbatim}
## Loading required package: splines
\end{verbatim}

\begin{Shaded}
\begin{Highlighting}[]
\CommentTok{# Muestra aleatoria de una función de densidad Rayleigh por medio de variables antitéticas}
\NormalTok{X <-}\StringTok{ }\OtherTok{NULL}\NormalTok{; Y <-}\StringTok{ }\OtherTok{NULL}\NormalTok{; n <-}\StringTok{ }\DecValTok{10000}\NormalTok{; sigma <-}\StringTok{ }\DecValTok{10}\NormalTok{; x <-}\StringTok{ }\OtherTok{NULL}\NormalTok{;}

\ControlFlowTok{for}\NormalTok{ (i }\ControlFlowTok{in} \DecValTok{1}\OperatorTok{:}\NormalTok{n)\{}
\NormalTok{  aux <-}\StringTok{ }\KeywordTok{runif}\NormalTok{(}\DecValTok{1}\NormalTok{,}\DecValTok{0}\NormalTok{,}\DecValTok{1}\NormalTok{);}
\NormalTok{  X[i] <-}\StringTok{ }\NormalTok{sigma }\OperatorTok{*}\StringTok{ }\KeywordTok{sqrt}\NormalTok{(}\KeywordTok{log}\NormalTok{(}\DecValTok{1}\OperatorTok{/}\NormalTok{(}\DecValTok{1}\OperatorTok{-}\NormalTok{aux)}\OperatorTok{^}\DecValTok{2}\NormalTok{));}
\NormalTok{  Y[i] <-}\StringTok{ }\NormalTok{sigma }\OperatorTok{*}\StringTok{ }\KeywordTok{sqrt}\NormalTok{(}\KeywordTok{log}\NormalTok{(}\DecValTok{1}\OperatorTok{/}\NormalTok{(aux)}\OperatorTok{^}\DecValTok{2}\NormalTok{));}
  
\NormalTok{  x[i] <-}\StringTok{ }\NormalTok{(X[i] }\OperatorTok{+}\StringTok{ }\NormalTok{Y[i])}\OperatorTok{/}\DecValTok{2}\NormalTok{;}
\NormalTok{\}}

\KeywordTok{hist}\NormalTok{(x, }\DataTypeTok{prob =}\NormalTok{ T, }\DataTypeTok{breaks=}\DecValTok{20}\NormalTok{, }\DataTypeTok{xlab=}\StringTok{"muestra"}\NormalTok{, }\DataTypeTok{ylab=}\StringTok{"densidad"}\NormalTok{)}
\KeywordTok{curve}\NormalTok{(}\KeywordTok{drayleigh}\NormalTok{(x, }\DataTypeTok{scale =}\NormalTok{ sigma, }\DataTypeTok{log =} \OtherTok{FALSE}\NormalTok{), }\DataTypeTok{from =} \DecValTok{0}\NormalTok{, }\DataTypeTok{to =} \DecValTok{100}\NormalTok{, }\DataTypeTok{add =}\NormalTok{ T, }\DataTypeTok{col =} \StringTok{"red"}\NormalTok{)}
\end{Highlighting}
\end{Shaded}

\includegraphics{Parcial-2-Simulación_files/figure-latex/unnamed-chunk-1-1.pdf}

Ahora, presento la varianza de la simulación de la muestra aleatoria por
medio de variables antitéticas

\begin{Shaded}
\begin{Highlighting}[]
\NormalTok{varAnti <-}\StringTok{ }\KeywordTok{var}\NormalTok{(x);}
\NormalTok{varAnti}
\end{Highlighting}
\end{Shaded}

\begin{verbatim}
## [1] 1.082794
\end{verbatim}

Por otro lado, simulo la otra muestra de variables aleatorias de la
misma densidad de Rayleigh con exactamente los mismos valores
\(n=10000\) y \(\sigma =10\), su parametro de escala. En esta ocasión,
simulo dos variables aleatorias distintas e independientes entre sí y
utilizo su combinación lineal de variables aleatorias dada por lo
siguiente \(\frac{U+V}{2}\). De esta manera, obtenemos el siguiente
histograma:

\begin{Shaded}
\begin{Highlighting}[]
\CommentTok{#Pregunta 1 parte 2}

\KeywordTok{library}\NormalTok{(VGAM)}
\CommentTok{# Muestra aleatoria de una función de densidad Rayleigh por medio de variables independientes}
\NormalTok{U <-}\StringTok{ }\OtherTok{NULL}\NormalTok{; V <-}\StringTok{ }\OtherTok{NULL}\NormalTok{; n <-}\StringTok{ }\DecValTok{10000}\NormalTok{; sigma <-}\StringTok{ }\DecValTok{10}\NormalTok{; x <-}\StringTok{ }\OtherTok{NULL}\NormalTok{;}


\ControlFlowTok{for}\NormalTok{ (i }\ControlFlowTok{in} \DecValTok{1}\OperatorTok{:}\NormalTok{n)\{}
\NormalTok{  U[i] <-}\StringTok{ }\NormalTok{sigma }\OperatorTok{*}\StringTok{ }\KeywordTok{sqrt}\NormalTok{(}\OperatorTok{-}\DecValTok{2} \OperatorTok{*}\StringTok{ }\KeywordTok{log}\NormalTok{(}\KeywordTok{runif}\NormalTok{(}\DecValTok{1}\NormalTok{,}\DecValTok{0}\NormalTok{,}\DecValTok{1}\NormalTok{)));}
\NormalTok{  V[i] <-}\StringTok{ }\NormalTok{sigma }\OperatorTok{*}\StringTok{ }\KeywordTok{sqrt}\NormalTok{(}\OperatorTok{-}\DecValTok{2} \OperatorTok{*}\StringTok{ }\KeywordTok{log}\NormalTok{(}\KeywordTok{runif}\NormalTok{(}\DecValTok{1}\NormalTok{,}\DecValTok{0}\NormalTok{,}\DecValTok{1}\NormalTok{)));}
\NormalTok{\}}
\NormalTok{y <-}\StringTok{ }\NormalTok{(U}\OperatorTok{+}\NormalTok{V)}\OperatorTok{/}\DecValTok{2}\NormalTok{;}

\KeywordTok{hist}\NormalTok{(y, }\DataTypeTok{prob =}\NormalTok{ T, }\DataTypeTok{breaks=}\DecValTok{20}\NormalTok{, }\DataTypeTok{xlab=}\StringTok{"muestra"}\NormalTok{, }\DataTypeTok{ylab=}\StringTok{"densidad"}\NormalTok{)}
\KeywordTok{curve}\NormalTok{(}\KeywordTok{drayleigh}\NormalTok{(x, }\DataTypeTok{scale =}\NormalTok{ sigma, }\DataTypeTok{log =} \OtherTok{FALSE}\NormalTok{), }\DataTypeTok{from =} \DecValTok{0}\NormalTok{, }\DataTypeTok{to =} \DecValTok{100}\NormalTok{, }\DataTypeTok{add =}\NormalTok{ T, }\DataTypeTok{col =} \StringTok{"green"}\NormalTok{)}
\end{Highlighting}
\end{Shaded}

\includegraphics{Parcial-2-Simulación_files/figure-latex/unnamed-chunk-3-1.pdf}

El histograma anterior resulta ser diferente al primero que se produjo
por medio de variables antitéticas. Sin embargo, la varianza producida
en esta ocasión es mayor, ya que son dos variables distintas que se
utilizan para obtener su información. Aunque pueda simular una
aproximación cercana a la distribución, resulta que los valores son
dispersos

\begin{Shaded}
\begin{Highlighting}[]
\NormalTok{varNormal <-}\StringTok{ }\KeywordTok{var}\NormalTok{(y);}
\NormalTok{varNormal}
\end{Highlighting}
\end{Shaded}

\begin{verbatim}
## [1] 21.20699
\end{verbatim}

Concluimos que el método de variables antitéticas reducen en gran
porcentaje la varianza de la simulación de la función de densidad. La
comparación de varianzas se hace con el supuesto de que la varianza por
medio de variables antitéticas es el \(100\)\%.

\begin{Shaded}
\begin{Highlighting}[]
\NormalTok{porcentajeVarianza <-}\StringTok{ }\NormalTok{(varAnti }\OperatorTok{*}\StringTok{ }\DecValTok{100}\NormalTok{)}\OperatorTok{/}\NormalTok{varNormal;}
\NormalTok{porcentajeVarianza}
\end{Highlighting}
\end{Shaded}

\begin{verbatim}
## [1] 5.105835
\end{verbatim}

Con lo anterior, se traduce que la varianza antitética es cercana a la
fracción \(\frac{1}{20}=0.05\) de la varianza obtenida por medio de
variables independientes entre sí.

\hypertarget{pregunta-5}{%
\subsection{Pregunta 5}\label{pregunta-5}}

En este problema simulo una cadena de Markov que simula los ataques a
los puertos de unas computadoras. Los puertos son los siguientes: 80,
134, 139, 445, -1 (este último para modelar el puerto vacío).

\[\begin{equation}
    A = \begin{bmatrix}
        0 & 0 & 0 & 0 & 1 \\
        0 & 8/13 & 3/13 & 1/13 & 1/13 \\
        1/16 & 3/16 & 3/8 & 1/4 & 1/8 \\
        0 & 1/11 & 4/11 & 5/11 & 1/11 \\
        0 & 1/8 & 1/2 & 1/8 & 1/4 \\
        \end{bmatrix}
\end{equation}\]

Ahora se simula el proceso de Markov después de \(100\) iteraciones para
obtener una nueva matriz e interpretarla

\begin{verbatim}
## Warning: package 'markovchain' was built under R version 3.6.3
\end{verbatim}

\begin{verbatim}
## Package:  markovchain
## Version:  0.8.5-2
## Date:     2020-09-07
## BugReport: https://github.com/spedygiorgio/markovchain/issues
\end{verbatim}

\begin{verbatim}
##              80       134       139       445     -1
## [1,] 0.02146667 0.2669333 0.3434667 0.2273333 0.1408
\end{verbatim}

Después de hacer la simulación, también busco cuál es la distribución
estacionaria, a partir del espacio de estados de la cadena de Markov.

\begin{Shaded}
\begin{Highlighting}[]
\NormalTok{distribucionEstacionaria <-}\StringTok{ }\KeywordTok{steadyStates}\NormalTok{(cadenaAtaques); }
\NormalTok{distribucionEstacionaria;}
\end{Highlighting}
\end{Shaded}

\begin{verbatim}
##              80       134       139       445     -1
## [1,] 0.02146667 0.2669333 0.3434667 0.2273333 0.1408
\end{verbatim}

\hypertarget{pregunta-3}{%
\subsection{Pregunta 3}\label{pregunta-3}}

La siguiente pregunta se escribió a mano y se adjunta en la siguiente
página.

\end{document}
